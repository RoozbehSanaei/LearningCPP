\documentclass[11pt]{article}

\usepackage[margin=1in]{geometry}
\usepackage{booktabs}
\usepackage{siunitx}
\usepackage{amsmath}
\usepackage{hyperref}

\sisetup{
  round-mode=places,
  round-precision=4
}

\title{Synthetic Planar Calibration Evaluation Across Varying Intrinsic Models}
\author{}
\date{}

\begin{document}
\maketitle

\section{Objective}
Quantify parameter recovery and reprojection performance of a planar calibration pipeline under controlled conditions by generating multiple synthetic datasets with known camera models, then estimating intrinsics, distortion, and per-view poses from noisy 2D measurements. The evaluation focuses on stability as the intrinsic matrix varies across runs.

\section{Experimental Design}

\subsection{Fixed configuration (shared across runs)}
\begin{itemize}
  \item Image resolution: \SI{1280 x 720}{px}
  \item Calibration target: planar grid, $9 \times 6$ points $\Rightarrow$ \SI{54}{} points per image
  \item Number of images per dataset: \SI{6}{} 
  \item Measurement noise: i.i.d.\ Gaussian noise, $\sigma=\SI{0.5}{px}$ applied independently to each observed $(u,v)$
  \item Lens distortion (held constant across runs): $k_1=-0.12$, $k_2=0.018$, $p_1=0.0012$, $p_2=-0.0007$
\end{itemize}

\subsection{Varied configuration (changed per run)}
For each of \SI{10}{} independent datasets, a distinct intrinsic model was sampled:
\begin{itemize}
  \item $f_x \sim \mathrm{Uniform}(600,1200)$
  \item $f_y = f_x \cdot \mathrm{Uniform}(0.94,1.06)$
  \item $c_x = W/2 + \mathrm{Uniform}(-50,+50)$
  \item $c_y = H/2 + \mathrm{Uniform}(-30,+30)$
\end{itemize}

\section{Data Generation Procedure}
For each dataset:
\begin{enumerate}
  \item Sample a ground-truth intrinsic model $(f_x,f_y,c_x,c_y)$ from the distributions above.
  \item Generate \SI{6}{} camera poses (one per image) with randomized rotation and translation, constrained to keep the planar target within image bounds (a fixed margin was applied).
  \item Project the planar points to pixel coordinates using the sampled intrinsics, the fixed distortion model, and the sampled pose.
  \item Add Gaussian pixel noise ($\sigma=\SI{0.5}{px}$) to all projected measurements to obtain the synthetic observations.
\end{enumerate}

\section{Calibration Pipeline}
For each dataset, parameter estimation proceeds as follows:
\begin{enumerate}
  \item \textbf{Per-image homography estimation:} estimate planar homographies via normalized DLT (Direct Linear Transform) from correspondences $(X,Y)\rightarrow(u,v)$.
  \item \textbf{Intrinsic initialization (planar self-calibration):} solve Zhang constraints from the set of homographies to obtain initial estimates of $f_x,f_y,c_x,c_y$ (skew fixed to zero in this implementation).
  \item \textbf{Pose initialization:} recover per-image rotation and translation by decomposing each homography using the intrinsic estimate, followed by orthonormalization (projection onto $\mathrm{SO}(3)$).
  \item \textbf{Nonlinear refinement (reprojection-error minimization):} jointly refine intrinsics, distortion, and poses via staged Levenberg--Marquardt:
  \begin{itemize}
    \item Phase A: intrinsics + poses (distortion held at zero)
    \item Phase B: enable $(k_1,k_2)$ + intrinsics + poses
    \item Phase C: enable $(k_1,k_2,p_1,p_2)$ + intrinsics + poses
  \end{itemize}
\end{enumerate}

\section{Evaluation Metrics}

\subsection{Reprojection RMSE (pixels)}
\begin{itemize}
  \item \textbf{Noise-floor RMSE:} RMSE obtained when projecting with ground-truth parameters onto noisy observations (lower bound under the chosen noise model).
  \item \textbf{Initialization RMSE:} RMSE using the homography/Zhang initialization (distortion fixed to zero).
  \item \textbf{Final RMSE:} RMSE after nonlinear refinement.
\end{itemize}

\subsection{Parameter recovery error}
\begin{itemize}
  \item Intrinsic error computed as estimate minus ground truth.
  \item Reported as MAE (mean absolute error) for each intrinsic component; relative MAE for $f_x$ and $f_y$; and MAE for distortion parameters.
\end{itemize}

\section{Results (10 datasets)}

\subsection{Aggregate reprojection performance (mean $\pm$ std, pixels)}
\begin{itemize}
  \item Noise-floor RMSE: $0.7154 \pm 0.0196$
  \item Initialization RMSE: $1.6070 \pm 0.4392$
  \item Final RMSE: $0.7463 \pm 0.0971$
  \item Final / noise-floor ratio: $1.0437 \pm 0.1402$
\end{itemize}
Interpretation: the refined solution is, on average, within approximately \SI{4.4}{\percent} of the irreducible error induced by the measurement noise.

\subsection{Intrinsic recovery error (estimate minus ground truth)}
\begin{itemize}
  \item $f_x$: MAE $19.99$ px, relative MAE $0.0262$ ($\approx \SI{2.62}{\percent}$)
  \item $f_y$: MAE $21.22$ px, relative MAE $0.0273$ ($\approx \SI{2.73}{\percent}$)
  \item $c_x$: MAE $33.32$ px
  \item $c_y$: MAE $14.15$ px
\end{itemize}

\subsection{Distortion recovery error (estimate minus ground truth)}
\begin{itemize}
  \item $k_1$: MAE $0.04933$
  \item $k_2$: MAE $0.14808$
  \item $p_1$: MAE $0.001678$
  \item $p_2$: MAE $0.002887$
\end{itemize}

\section{Per-run Summary}
\begin{table}[h!]
\centering
\begin{tabular}{rrrrrrr}
\toprule
run & $f_{x,\mathrm{gt}}$ & $f_{y,\mathrm{gt}}$ & $c_{x,\mathrm{gt}}$ & $c_{y,\mathrm{gt}}$ & $\mathrm{RMSE}_{\mathrm{floor}}$ & $\mathrm{RMSE}_{\mathrm{final}}$ \\
\midrule
1  & 633.2  & 622.2  & 670.4 & 331.2 & 0.6967 & 0.6779 \\
2  & 738.9  & 763.0  & 596.6 & 365.0 & 0.7112 & 0.7127 \\
3  & 870.9  & 898.8  & 616.0 & 339.9 & 0.7040 & 0.6881 \\
4  & 836.2  & 799.0  & 640.4 & 347.4 & 0.7229 & 0.6949 \\
5  & 650.2  & 648.3  & 605.7 & 368.2 & 0.7142 & 0.6902 \\
6  & 736.2  & 739.0  & 615.8 & 364.4 & 0.7085 & 0.6773 \\
7  & 1117.4 & 1110.3 & 639.4 & 356.7 & 0.7285 & 0.7532 \\
8  & 1149.8 & 1080.9 & 601.6 & 363.6 & 0.7546 & 0.8899 \\
9  & 931.2  & 897.1  & 664.5 & 370.3 & 0.6847 & 0.9559 \\
10 & 746.2  & 761.6  & 652.4 & 347.1 & 0.7293 & 0.7230 \\
\bottomrule
\end{tabular}
\end{table}

\section{Discussion}
\begin{enumerate}
  \item \textbf{Convergence to the noise floor:} the final RMSE tracks the noise floor closely on average, indicating that the reprojection objective is being minimized effectively under typical sampled intrinsics.
  \item \textbf{Outlier runs:} runs 8 and 9 exhibit higher final RMSE relative to their noise floor, consistent with the sensitivity of planar calibration to pose diversity and parameter coupling (intrinsics--distortion--pose trade-offs).
  \item \textbf{Principal point variability:} $c_x,c_y$ show larger absolute MAE than focal lengths, which is common when view diversity is limited and distortion competes with principal point shifts.
\end{enumerate}

\section{Limitations}
\begin{itemize}
  \item \textbf{Planar degeneracies and coupling:} planar targets inherently provide weaker constraints than 3D calibration rigs; parameter coupling can produce multiple near-equivalent solutions under noise.
  \item \textbf{Numeric Jacobian:} finite-difference Jacobians are serviceable but less stable and less efficient than analytic derivatives.
  \item \textbf{Limited number of images per dataset (6):} this reduces robustness, particularly for separating distortion from intrinsics.
\end{itemize}

\section{Recommendations}
\begin{itemize}
  \item Increase the number of images per dataset (e.g., 15--30) with stronger tilt and roll diversity.
  \item Ensure broader spatial coverage of the target across the image (near corners and edges).
  \item Replace numeric Jacobians with analytic Jacobians for the projection + distortion model.
  \item Add mild regularization or priors (e.g., constrain $c_x,c_y$ near the image center when appropriate).
\end{itemize}

\end{document}
